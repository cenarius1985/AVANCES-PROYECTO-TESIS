\documentclass[11pt,a4paper]{article}

% --- Paquetes Básicos ---
\usepackage[utf8]{inputenc}
\usepackage[spanish,es-tabla]{babel}
\usepackage[T1]{fontenc}
\usepackage{geometry}
\geometry{top=2.5cm, bottom=2.5cm, left=3cm, right=3cm}

% --- Estética y Colores ---
\usepackage{xcolor}
\definecolor{UVBlue}{RGB}{0, 51, 102} % Azul corporativo UV sugerido
\usepackage{titlesec}
\titleformat{\section}{\color{UVBlue}\normalfont\Large\bfseries}{\thesection}{1em}{}
\titleformat{\subsection}{\color{UVBlue}\normalfont\large\bfseries}{\thesubsection}{1em}{}

% --- Gráficos ---
\usepackage{graphicx}
\usepackage{fancyhdr}
\usepackage{lastpage}

% --- Tablas y Cronogramas ---
\usepackage{booktabs}
\usepackage{tabularx}
\usepackage{pgfgantt}
\usepackage{tikz}

% --- Enlaces ---
\usepackage[colorlinks=true, linkcolor=UVBlue, citecolor=UVBlue, urlcolor=blue]{hyperref}

% --- Configuración de Encabezado y Pie de Página ---
\pagestyle{fancy}
\fancyhf{}
\fancyhead[L]{\includegraphics[height=1.2cm]{Figures/uv_logo.png}}
\fancyhead[R]{\includegraphics[height=1.2cm]{Figures/uv_acreditacion.png}}
\fancyfoot[C]{\thepage\ de \pageref{LastPage}}
\fancyfoot[L]{\footnotesize Doctorado en Ingeniería Informática Aplicada}
\renewcommand{\headrulewidth}{0.4pt}

% --- Datos del Documento ---
\title{
    \vspace{0.5cm}
    \textcolor{UVBlue}{Informe de Avances: Proyecto de Tesis Doctoral} \\
    \large{Análisis de Hueso Cortical mediante MRI 0.55T y Ultrasonido BDAT}
}
\author{
    \textbf{Fernando Ramírez Sarmiento} \\
    \small{Candidato a Doctor en Ingeniería Informática Aplicada} \\
    \small{Universidad de Valparaíso}
}
\date{\today}

\begin{document}

\thispagestyle{fancy} % Forzar encabezado en la primera página
\maketitle

\begin{tikzpicture}[remember picture, overlay]
    \node[anchor=north west, xshift=3cm, yshift=-1.5cm] at (current page.north west) 
        {\includegraphics[height=1.2cm]{Figures/uv_logo.png}};
    \node[anchor=north east, xshift=-3cm, yshift=-1.5cm] at (current page.north east) 
        {\includegraphics[height=1.2cm]{Figures/uv_acreditacion.png}};
\end{tikzpicture}

\vspace{0.5cm}

\section{Resumen Ejecutivo}
El estado de avance del proyecto de tesis doctoral, cuyo eje central es la validación de secuencias de Tiempo de Eco Ultracorto (UTE) en resonancia magnética de bajo campo (0.55 T), contrastando los resultados obtenidos con ultrasonido de transmisión axial bidireccional (BDAT).

Se han logrado avances significativos en la confección de la secuencia, mejoras en la reconstrucción y en la descripcion del proceso de relajación transversal, finalizando con la consolidación de la metodología de adquisición en resonancia magnética.

Sin embargo, existen puntos críticos, principalmente retrasos en el cronograma. Las mediciones en los 16 sujetos de prueba programadas para los estudios de RM y BDAT se encuentran retrasadas en 2 meses con 0 avances. Se pretende retomar de inmediato estas actividades entre marzo y abril de 2026, con el análisis de los resultados planificado para mayo y el término de la redacción del artículo científico en julio para la búsqueda de su publicación. La defensa de la tesis doctoral se estima para marzo de 2027.

\section{Estado de Avance}
\subsection{Hitos Logrados}
\begin{itemize}
    \item \textbf{Diseño de Secuencia UTE:} Implementación exitosa de una secuencia radial 2D optimizada para Siemens Magnetom Free.Max (0.55 T).
    \item \textbf{Validación en Fantomas:} Confirmación de la fidelidad cuantitativa de $T_2^*$ con errores menores al 3\%.
    \item \textbf{Estudio Preliminar in vivo:} Adquisición y procesamiento de datos en voluntarios iniciales, demostrando la factibilidad de separar agua ligada y libre.
    \item \textbf{Integración BDAT:} Estandarización del protocolo de adquisición multimodal en mid-tibia.
\end{itemize}

\subsection{Producción Científica}
Actualmente se trabaja en el manuscrito \textit{"Quantitative Cortical Bone Assessment using Low-Field UTE MRI and BDAT Ultrasound"}, con objetivo de envío a revista Q1.


\section{Resultados Preliminares}

El análisis de la relajación transversal $T_2^*$ mediante el modelo bi-exponencial permitió la identificación y separación de las dos fracciones de agua características del hueso cortical: la componente rápida ($T_{2,s}^*$), asociada al agua ligada al colágeno, y la componente larga ($T_{2,l}^*$), correspondiente al agua de los poros óseos (ver Figura~\ref{fig:result_images}).

Un desafío crítico abordado fue la contaminación por señal proveniente de la médula ósea adyacente, cuyo espectro de relajación se solapa parcialmente con la cola del decaimiento de la señal ósea. Para mitigar este efecto de volumen parcial, se incorporó la estimación de la componente de relajación medular en el modelo de ajuste, utilizándola como límite superior para restringir el rango de decaimiento atribuible exclusivamente al hueso. Esto permitió aislar con mayor precisión la señal cortical, minimizando la sobreestimación de la fracción de agua de poro ($PW_{frac}$) causada por la infiltración de lípidos medulares en la región de interés (ROI).

Como se observa en los gráficos de la Figura~\ref{fig:result_images}, la señal ajustada (línea continua) sigue fielmente los datos experimentales (puntos), validando la capacidad del protocolo a 0.55 T para resolver componentes de vida media ultracorta ($< 1$ ms) en presencia de especies de relajación más lenta.

\begin{figure}[htbp]
    \centering
    \begin{minipage}{0.48\textwidth}
        \centering
        \includegraphics[width=\linewidth]{Figures/resultados/t2_star_rician_correlaciones_cortical_20260218_113959_sujeto1_40_anios.png}
        \caption{Sujeto 1 (40 años)}
    \end{minipage}\hfill
    \begin{minipage}{0.48\textwidth}
        \centering
        \includegraphics[width=\linewidth]{Figures/resultados/t2_star_rician_correlaciones_cortical_20260218_114735_sujeto2_42_anios.png}
        \caption{Sujeto 2 (42 años)}
    \end{minipage}
    \caption{Curvas de decaimiento $T_2^*$ en hueso cortical mostrando la separación de componentes rápida (agua ligada) y larga (agua de poro). Se evidencia la contribución de la señal medular en los tiempos de eco tardíos, la cual es modelada para evitar sesgos en la cuantificación ósea.}
    \label{fig:result_images}
\end{figure}

\section{Problemas y Desafíos}
Se identifican los siguientes puntos críticos para la etapa final:
\begin{itemize}
    \item \textbf{Tamaño de Muestra:} Necesidad de completar el reclutamiento de 16 voluntarios sanos para alcanzar significancia estadística.
    \item \textbf{Procesamiento de Datos:} Optimización del algoritmo de separación grasa-agua (hmrGC) para el campo de 0.55 T.
    \item \textbf{Artefactos de Movimiento:} Minimización del impacto del tiempo de adquisición (aprox. 54 min) en la calidad de imagen.
    \item \textbf{Análisis Estadístico Multimodal:} Validación final de la correlación entre los biomarcadores de MRI ($PW_{frac}$, $BW_{frac}$) y los parámetros de ultrasonido BDAT ($V_{FAS}$, $Ct.Th$, $Ct.Po$).
\end{itemize}


\section{Planificación y Cronograma}
\begin{center}
\begin{ganttchart}[
    hgrid,
    vgrid,
    x unit=0.8cm,
    y unit title=0.7cm,
    title/.style={draw=none, fill=UVBlue!20},
    title label font=\bfseries\small,
    bar/.style={fill=UVBlue},
    bar height=0.6,
    progress label text={},
    group label font=\bfseries\small,
    milestone label font=\itshape\small
]{1}{12}
    \gantttitle{Meses Restantes (2026)}{12} \\
    \gantttitlelist{1,...,12}{1} \\
    \ganttbar{Finalizar Reclutamiento}{1}{3} \\
    \ganttbar{Procesamiento Final}{3}{6} \\
    \ganttbar{Escritura Tesis}{6}{10} \\
    \ganttmilestone{Envío Paper Q1}{8} \\
    \ganttbar{Revisión y Formato}{10}{12} \\
    \ganttmilestone{Entrega Final}{12}
\end{ganttchart}
\end{center}


\section{Trabajo Pendiente y Siguientes Pasos}
\section{Trabajo Pendiente y Siguientes Pasos}
Para la culminación del proyecto de tesis doctoral, se han definido los siguientes hitos técnicos y administrativos que restan por completar:

\begin{itemize}
    \item \textbf{Finalización del Reclutamiento:} Completar la adquisición de datos en los voluntarios restantes hasta alcanzar el $N=16$ planificado.
    \item \textbf{Desarrollo de Suite de Análisis Dockerizada:} Empaquetar los algoritmos de secuencia UTE, reconstrucción y análisis de relajación transversal en una plataforma unificada y contenerizada (Docker). El objetivo es desplegar estos servicios como una API REST de uso local, abstraer la complejidad de la implementación y facilitar su consumo por usuarios finales para el análisis de hueso cortical y tejidos con tiempos de relajación ultracortos.
    \item \textbf{Consolidación de Resultados Multimodales:} Ejecutar el análisis de correlación cruzada entre los hallazgos de MRI y las mediciones de ultrasonido BDAT.
    \item \textbf{Escritura y Revisión de Manuscritos:} Finalizar la redacción de los artículos para revistas Q1 y la integración de capítulos en el documento final de tesis.
    \item \textbf{Trámites Administrativos de Defensa:} Cumplir con los plazos de pre-defensa y defensa pública establecidos por el programa de doctorado de la Universidad de Valparaíso para el segundo semestre de 2026.
\end{itemize}


\section{Propuesta de Defensa y Finalización}
Se estima la entrega definitiva del manuscrito de tesis para el último trimestre de 2026. 

\paragraph{Hitos para la Finalización:}
\begin{itemize}
    \item \textbf{Cierre de Base de Datos:} Marzo 2026.
    \item \textbf{Pre-defensa Interna:} Agosto 2026.
    \item \textbf{Defensa Pública:} Diciembre 2026.
\end{itemize}


\end{document}
