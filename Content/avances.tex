\subsection{Hitos Logrados}
\begin{itemize}
    \item \textbf{Diseño de Secuencia UTE:} Implementación exitosa de una secuencia radial 2D optimizada para Siemens Magnetom Free.Max (0.55 T).
    \item \textbf{Reconstrucción radial SENSE y Compressed Sensing:} Implementación exitosa de SENSE y Compressed Sensing, estandariza la intensidad para reflejar el proceso de relajación transversal.
    \item \textbf{Validación en Fantomas:} Confirmación de la fidelidad cuantitativa de $T_2^*$ con errores menores al 3\%.
    \item \textbf{Modelado de Relajación Transversal:} Implementación de ajustes mono y bi-exponenciales adaptados para hueso cortical que integran la compensación de ruido Rician en la formulación, abordando el rápido decaimiento de la señal y mitigando artefactos de grasa medular mediante estrategias para componentes de relajación lenta.
    \item \textbf{Estudio Preliminar in vivo:} Adquisición y procesamiento de datos en voluntarios iniciales, demostrando la factibilidad de separar agua ligada y libre.

\end{itemize}

\subsection{Producción Científica}
Actualmente se trabaja en el manuscrito \textit{"Bi-Component Quantification of Cortical Bone Water Fractions Using Low-Field UTE MRI: Comparison with BDAT Ultrasound"}, con objetivo de publicar el estudio.
